% Resume in Latex
% Author : Parth Kharade
% chktex-file 8
% chktex-file 1

\documentclass[11pt]{article}
\usepackage{latexsym}
\usepackage[margin=1cm]{geometry}
\usepackage{titlesec}
\usepackage[usenames,dvipsnames]{color}
\usepackage{enumitem}
\usepackage[hidelinks]{hyperref}
\usepackage{tabularx}
\usepackage{fontawesome5}
\pagenumbering{gobble}
% \raggedbottom
\raggedright
\setlength{\tabcolsep}{0in}

% Sections formatting
\titleformat{\section}{
  \vspace{-4pt}\raggedright\large\bfseries
}{}{0em}{}[\color{black}\titlerule\vspace{-5pt}]

% Ensure that generate pdf is machine readable/ATS parsable
\input{glyphtounicode}
\pdfgentounicode=1

%-------------------------
% Custom commands

\newcommand{\resumeSubheadingEdu}[4]{
  \vspace{-2pt}\item
    \begin{tabular*}{1.0\textwidth}[t]{l@{\extracolsep{\fill}}r}
       {\large #1}&{\large #2} \\
       {#3} &{ #4} \\
    \end{tabular*}\vspace{-1pt}
}

\newcommand{\resumeSubheading}[3]{
  \vspace{-2pt}\item
    \begin{tabular*}{1.0\textwidth}[t]{l@{\extracolsep{\fill}}r}
      {\large \textbf{#1} \textbar{} \emph{#2}}&{\large #3} \\
    \end{tabular*}\vspace{-10pt}
}


\newcommand{\resumeSubheadingP}[2]{
  \vspace{-2pt}\item
    \begin{tabular*}{1.0\textwidth}[t]{l@{\extracolsep{\fill}}r}
       \textbf{{\large #1}}&{#2} \\
    \end{tabular*}\vspace{-1pt}
}


\newcommand{\resumeSubHeadingListStart}{\begin{itemize}[leftmargin=0.0in, label={}]}
\newcommand{\resumeSubHeadingListEnd}{\end{itemize}}

\begin{document}
\setlength{\footskip}{4pt}
\begin{center}
  {\huge \scshape Parth Kharade} \\ \vspace{1pt}
  \small \raisebox{-0.1\height}\faPhone\ 303-725-7695{\hspace{0.1cm}}~\href{mailto:parthkharade99@gmail.com}{\raisebox{-0.2\height}\faEnvelope\ \underline{parthkharade99@gmail.com}}~{\hspace{0.1cm}}\href{https://www.linkedin.com/in/parth-k-081287184/}{\raisebox{-0.2\height}\faLinkedin\ \underline{linkedin.com/in/parthkharade}}
  \vspace{-10pt}
\end{center}


%-----------EDUCATION-----------

\section {Education}
\resumeSubHeadingListStart
\resumeSubheadingEdu
{University of Colorado Boulder}{May 2025}
{Master's in Embedded Systems Engineering}{3.8/4.0}
\vspace{-0.1in}
\resumeSubheadingEdu
{Birla Institute of Technology and Science, Pilani}{May 2021}
{Bachelor's in Electrical and Electronics Engineering}{7.9/10.0}
\resumeSubHeadingListEnd
\vspace{-16pt}
\section{Skills and Coursework}
\vspace{-4mm}
\begin{table}[!htb]
  \begin{tabular} { m{3.1cm} | m{15cm} }
    % {Micro-controllers}   & {\: STM32F4, TI-MSP430, TI-C2000, 8051}                  \\
    {Languages}           & {\: C, 8051 Assembly,ARM Assembly, Python, Verilog, VHDL, Bash}  \\
    {Tools/Software}      & {\: Make, Bazel, Git, Jira, GTest, Quartus, Vivado, Modelsim}     \\
    {Peripherals}         & {\: GPIO, SPI, UART, I2C, ADC, DAC}                 \\
    {Hardware}            & {\: Layout and Schematic Capture, Circuit Design, EAGLE, KiCad, LTSpice}                        \\
    {Debugging}           & {\: Oscilloscopes, Logic Analyzer, JTAG}\\
    % {Coursework}          & {\: Embedded System Design, Embedded Linux Design, Real-Time Embedded Systems}                \\
  \end{tabular}
\end{table}
\vspace{-16pt}
%-----------EXPERIENCE-----------
\section{Experience}
\resumeSubHeadingListStart
\resumeSubheading
{Carbon3D}{Software Engineering Intern}{May 2024 - Aug 2024}
\vspace{-0.3cm}
\begin{itemize}[leftmargin=0.3in]\vspace{-0.3em}\setlength{\itemsep}{0pt}
  \item[$\bullet$]Optimized internal state machines for Carbon3D printers by developing rigorous GoogleTest cases and improving application code, eliminating unintended state transitions and improving system stability.
  \item[$\bullet$] Built a firmware update manager in Python to provide a consistent interface for managing printer firmware. 
\end{itemize}
\vspace{-0.2cm}
%---------------------------------------------------------------------%
\resumeSubheading
{WCB Robotics Pvt Ltd}{Design Engineer}{Jul 2021 - May 2023}
\vspace{-0.4cm}
\begin{itemize}[leftmargin=0.01in]\setlength\itemsep{-3pt}
  \item[]  \textbf{\emph{RF Transceiver}}
    \begin{itemize}\setlength{\itemsep}{0pt}\setlength{\parskip}{0pt}\vspace{-0.1em}
      \item[$\bullet$] Developed an SPI driver for the CC2500 transceiver that simplified access to core functionality. Enhanced noise immunity of the system by implementing FHSS and antenna diversity in firmware.
      % \item[$\bullet$] Designed a impedance-matched PCB for CC2500, achieving a line of sight range of 850 meters and reliable communication on high-rise buildings.
      % \item[$\bullet$] Designed a high-range wireless transceiver with an impedance-matched PCB for a facade cleaning robot, achieving a line of sight range of 850 meters and reliable communication on high-rise buildings.
       \item[$\bullet$] Wrote interrupt-driven code and utilized low power modes of the transceiver and the microcontroller to reduce power consumption and extend battery life of the hand-held radio controller.
      \item[$\bullet$] Worked in a cross-functional environment with mechanical and robotics teams to lead multiple board bring-ups define the software architecture, ensuring seamless integration.
    \end{itemize}
  \item[]  \textbf{\emph{Pressure and Orientation Sensing System}}
    \begin{itemize}\setlength{\itemsep}{0pt}\setlength{\parskip}{0pt}\vspace{-0.1em}
      \item[$\bullet$] Wrote a fault-tolerant I2C driver in C for the sensors to detect timeouts and report errors for timely remedial actions in a critical closed-loop control system, ensuring safe and reliable robot operation.
        % \item[$\bullet$] Collaborated with the mechanical team to develop a novel two-PCB (rigid +flex) arrangement, minimizing vibrational noise for reliable IMU operation.
        % \item[$\bullet$] This design was crucial to meet a tight-deadline in order to get the robot ready for an investor meeting.
   \end{itemize}


  % \item[]  \textbf{\emph{Lithium Ion Battery Charger}}
  %   \begin{itemize}\setlength{\itemsep}{0pt}\setlength{\parskip}{0pt}\vspace{-0.1em}
  %     \item[$\bullet$] Designed a compact Li-ion battery charger system by integrating a 4-layer PCB with the battery pack, allowing in-device charging without unplugging, while effectively managing thermal constraints. 
  %     \item[$\bullet$] Attained 250W/10A high-power charging using LT3763 IC, enabling a 21Ah battery to charged in 2.5 hours.
  %   \end{itemize}


  % \item[]  \textbf{\emph{Embedded Data Logger}}
  %   \begin{itemize}\setlength{\itemsep}{0pt}\setlength{\parskip}{0pt}\vspace{-0.1em}
  %     \item[$\bullet$] Interfaced a Winbond W25N NAND-Flash memory with a TMS570 microcontroller via SPI and wrote user-friendly APIs for efficient data storage and retrieval, enhancing data-logging capabilities.
  %       % \item[$\bullet$] Wrote a Python script for interpreting and processing retrieved data, improving data analysis and usability.
    % \end{itemize}
\end{itemize}
\vspace{-0.2cm}
%---------------------------------------------------------------------%
\resumeSubheading
{CU Boulder}{Teaching Assistant - Embedded System Design}{Jan 2024 - Dec 2024}
\vspace{-0.3cm}
\begin{itemize}[leftmargin=0.3in]\vspace{-0.3em}
  \item[$\bullet$]Mentored students in board bring-ups and debugging firmware issues. Conducted code reviews and collaborated with professors and peers to improve course structure and evaluations.
\end{itemize}
\vspace{-0.2cm}
\resumeSubHeadingListEnd
%---------------------------------------------------------------------%
\vspace{-14pt}

\section{Projects}
\resumeSubHeadingListStart

\resumeSubheadingP
{E-Paper Desk Clock \textbar{ } Bare-Metal Embedded \href{https://github.com/parthkharade/Eink-DeskClock}{\small{\textbar{ }\underline{Link} }}}{}
\begin{itemize}[leftmargin=0.3in]\setlength{\itemsep}{0pt}\setlength{\parskip}{0pt}\vspace{-0.2cm}
  \item[$\bullet$] Built a battery powered desk-clock using an E-Paper display. Wrote low-level peripheral drivers for STM32. 
  \item[$\bullet$] Developed a custom graphics library to allow the user to render bit-mapped content on the display. Implemented an MVC architecture to enable user interaction to set time and alarms.
  % \item[$\bullet$] Developed a custom graphics library  to allow the user to render bit mapped text and images on the display.
  % \item[$\bullet$] Implemented the Model-View-Controller design pattern using a state machine to allow the user to interact with the clock. The user could set date and time as well as configure 4 independent alarms.
\end{itemize}
\vspace{-0.2cm}


\resumeSubheadingP
{Audio Filter on FPGA \textbar{ } FPGA Design}{ }
\begin{itemize}[leftmargin=0.3in]\setlength{\itemsep}{0pt}\setlength{\parskip}{0pt}\vspace{-0.2cm}
  \item[$\bullet$] Developed a FIR filter in Verilog to reduce noise in audio signals sampled from a microphone.
  \item[$\bullet$] Designed and implemented a state-machine-based handshaking protocol to facilitate clock domain crossing between the audio codec and the FIR filter.
  \item[$\bullet$] Gained hands-on experience with key aspects of FPGA development, including timing closure, gate-level simulations, and soft-IP programming.
\end{itemize}
\vspace{-0.3cm}

% \resumeSubheadingP
% {NIOS II Custom Instruction Set \textbar{ } FPGA Design}{}
% \begin{itemize}[leftmargin=0.3in]\setlength{\itemsep}{0pt}\setlength{\parskip}{0pt}\vspace{-0.2cm}
%   \item[$\bullet$] Implemented custom instructions in Verilog for a NIOS II soft-processor to enhance arithmetic operations.
%   % \item[$\bullet$] Developed firmware for the soft-processor to use the timer module to benchmark execution times against standard C implementations.
%   \item[$\bullet$] Developed C firmware for the NIOS II soft processor to benchmark custom instructions against standard C implementations, validating performance improvements in an FPGA environment.
% \end{itemize}
% \vspace{-0.2cm}
%


\resumeSubheadingP
{Driver Assistance System \textbar{ } Real-Time Embedded Systems }{}
\begin{itemize}[leftmargin=0.3in]\setlength{\itemsep}{0pt}\setlength{\parskip}{0pt}\vspace{-0.2cm}
  \item[$\bullet$] Developed a real-time driver assistance system using RTLinux, integrating multi-core scheduling for synchronized frame capture, lane departure detection, and human detection, improving system response time.
  % \item[$\bullet$] Prototyped a driver assistance system using RTLinux. Synchronized multiple tasks such as frame capture, lane departure, human detection, running on multiple cores using cyclic rate monotonic schedulers and semaphores.
\end{itemize}
\vspace{-0.2cm}

% \resumeSubheadingP
% % {Electrical Power Subsystem of a Cube-Sat \href{https://github.com/parthkharade/MSP430F5529}{\small{\textbar{ }\underline{Link} }}}{}
% {Electrical Power Subsystem of a Cube-Sat \textbar{} Bare-Metal Embedded}{}
% \begin{itemize}[leftmargin=0.3in]\setlength{\itemsep}{0pt}\setlength{\parskip}{0pt}\vspace{-0.2cm}
%   \item[$\bullet$] Wrote I2C drivers for interfacing INA219 Power Monitor with the MSP430 microcontroller to implement Maximum Power Point Tracking using the Perturb and Observe algorithm.
% \end{itemize}
% \vspace{-0.2cm}
%
% \resumeSubheadingP
% {Spotify Dock \textbar{ } Embedded Linux \href{https://github.com/cu-ecen-aeld/final-project-parthkharade/wiki/Project-Overview}{\small{\textbar{ }\underline{Link} }}}{}
% \begin{itemize}[leftmargin=0.3in]\setlength{\itemsep}{0pt}\setlength{\parskip}{0pt}\vspace{-0.2cm}
%   \item[$\bullet$] Built a Spotify Dock, a device capable of displaying and controlling live Spotify playback
%     using bash scripts to interface with the Spotify API. Compiled a custom Buildroot image for RPi.
% \end{itemize}
% \vspace{-0.2cm}






% \resumeSubheadingP
% {MIPS Processor \textbar{ } Computer Architecture}{}
% \begin{itemize}[leftmargin=0.3in]\setlength{\itemsep}{0pt}\setlength{\parskip}{0pt}\vspace{-0.2cm}
%   \item[$\bullet$] Designed a 5-stage MIPS pipelined processor in Verilog capable of executing \emph{sub}, \emph{add}, \emph{or}, \emph{and} and \emph{beq} instructions. The pipeline was also able to detect and remedy structural, control and data hazards.
% \end{itemize}
% \vspace{-0.3cm}

% \resumeSubheadingP
% {Boulder Bash \textbar{ } Bluetooth Low Energy}{}
% \begin{itemize}[leftmargin=0.3in]\setlength{\itemsep}{0pt}\setlength{\parskip}{0pt}\vspace{-0.2cm}
%   \item[$\bullet$] Developed 'Boulder Bash', a retro arcade console game using Bluetooth Low Energy (BLE) with custom GATT services for data exchange between a energy-efficient wireless joystick and console with a dot matrix display. 
% \end{itemize}
% \vspace{-0.2cm}


% \resumeSubheadingP
% {Microchip AT89C51RC2 Board bring-up \href{https://drive.google.com/drive/folders/1rbtfpOdMc_ohnfu15VS3bPxNDx-BoWwA?usp=drive_link}{\small{\textbar{ }\underline{Link}}}}{August 2023 - November 2023}
% \begin{itemize}[leftmargin=0.3in]\setlength{\itemsep}{0pt}\setlength{\parskip}{0pt}\vspace{-0.2cm}
%   \item[$\bullet$] Built an AT89C51RC2 development board from scratch, interfacing various peripherals such as external RAM, LCD, GPIO expander, DAC, and ADC, and enabled in-system programming using a MAX232 IC.
%   % \item[$\bullet$] Wrote terminal-based, menu-driven, user-interactive programs to allow the user to interact with and configure on-chip as well as off-chip peripherals.
% \end{itemize}
% \vspace{-0.3cm}

\resumeSubHeadingListEnd
\end{document}
