%-------------------------
% Resume in Latex
% Author : Jake Gutierrez
% Based off of: https://github.com/sb2nov/resume
% License : MIT
%------------------------

\documentclass[a4,11pt]{article}

\usepackage{latexsym}
\usepackage[empty]{fullpage}
\usepackage{titlesec}
\usepackage{marvosym}
\usepackage[usenames,dvipsnames]{color}
\usepackage{verbatim}
\usepackage{enumitem}
\usepackage[hidelinks]{hyperref}
\usepackage{fancyhdr}
\usepackage[english]{babel}
\usepackage{tabularx}
\usepackage{fontawesome5}
\usepackage{multicol}
\setlength{\multicolsep}{-3.0pt}
\setlength{\columnsep}{-1pt}
\input{glyphtounicode}
\usepackage{array}

\pagestyle{fancy}
\fancyhf{} % clear all header and footer fields
\fancyfoot{}
\renewcommand{\headrulewidth}{0pt}
\renewcommand{\footrulewidth}{0pt}

% Adjust margins
\addtolength{\oddsidemargin}{-0.6in}
\addtolength{\evensidemargin}{-0.5in}
\addtolength{\textwidth}{1.19in}
\addtolength{\topmargin}{-.7in}
\addtolength{\textheight}{1.4in}

\urlstyle{same}

\raggedbottom
\raggedright
\setlength{\tabcolsep}{0in}

% Sections formatting
\titleformat{\section}{
  \vspace{-4pt}\scshape\raggedright\large\bfseries
}{}{0em}{}[\color{black}\titlerule \vspace{-5pt}]

% Ensure that generate pdf is machine readable/ATS parsable
\pdfgentounicode=1

%-------------------------
% Custom commands
\newcommand{\resumeItem}[1]{
  \item\small{
    {#1 \vspace{-2pt}}
  }
}

\newcommand{\classesList}[4]{
    \item\small{
        {#1 #2 #3 #4 \vspace{-2pt}}
  }
}

\newcommand{\resumeSubheading}[4]{
  \vspace{-2pt}\item
    \begin{tabular*}{1.0\textwidth}[t]{l@{\extracolsep{\fill}}r}
       {\large \textbf{#1}} &  {\large #2} \\
      \textbf{#3} & { #4} \\
    \end{tabular*}\vspace{-1pt}
}

\newcommand{\resumeSubheadingP}[2]{
  \vspace{-2pt}\item
    \begin{tabular*}{1.0\textwidth}[t]{l@{\extracolsep{\fill}}r}
       \textbf{{\large #1}} &  {#2} \\
    \end{tabular*}\vspace{-1pt}
}

\newcommand{\resumeSubSubheading}[2]{
    \item
    \begin{tabular*}{0.97\textwidth}{l@{\extracolsep{\fill}}r}
      \textit{\small#1} & \textit{ #2} \\
    \end{tabular*}\vspace{-7pt}
}

\newcommand{\resumeProjectHeading}[2]{
    \item
    \begin{tabular*}{1.001\textwidth}{l@{\extracolsep{\fill}}r}
      \small#1 &  {\small #2}\\
    \end{tabular*}\vspace{-7pt}
}

\newcommand{\resumeSubItem}[1]{\resumeItem{#1}\vspace{-4pt}}

\renewcommand\labelitemi{$\vcenter{\hbox{\tiny$\bullet$}}$}
\renewcommand\labelitemii{$\vcenter{\hbox{\tiny$\bullet$}}$}

\newcommand{\resumeSubHeadingListStart}{\begin{itemize}[leftmargin=0.0in, label={}]}
\newcommand{\resumeSubHeadingListEnd}{\end{itemize}}
\newcommand{\resumeItemListStart}{\begin{itemize}}
\newcommand{\resumeItemListEnd}{\end{itemize}\vspace{-5pt}}
\newcommand{\project}[3]{\texorpdfstring{$\bullet$}{}\hspace{4pt} {#1}\texorpdfstring{\hfill}{}\emph{#2}\texorpdfstring{\\}{}\hspace*{8pt}\emph{#3}}

\begin{document}

\begin{center}
    {\Huge \scshape Parth Kharade} \\ \vspace{1pt}
    \small \raisebox{-0.1\height}\faPhone\ 303-725-7695 ~ \href{mailto:parthkharade99@gmail.com}{\raisebox{-0.2\height}\faEnvelope\  \underline{parthkharade99@gmail.com}} ~ 
    \href{https://www.linkedin.com/in/parth-k-081287184/}{\raisebox{-0.2\height}\faLinkedin\ \underline{linkedin.com/in/parthkharade}}  ~
    \vspace{-16pt}
\end{center}


%-----------EDUCATION-----------
\section{Education}
  \resumeSubHeadingListStart
    \resumeSubheading
      {University of Colorado, Boulder}{August 2023 - May 2025}
      {Master's in Embedded Systems Engineering}{4.00/4.00}
    \resumeSubheading
      {Birla Institute of Technology and Science, Pilani}{August 2017 - May 2021}
      {Bachelor's in Electrical and Electronics}{7.85/10.0}
  \resumeSubHeadingListEnd
 \vspace{-16pt}
 \section{Skills}
 \vspace{-4mm}
\begin{table}[!htb]
    \begin{tabular}{ m{3.1cm} | m{16cm} } 
      {Micro-controllers}&{\: STM32 (Cortex-M4F), NXP(Cortex-M0+), MSP430, TMS320, 8051, Atmel} \\
      {Firmware}&{\: C, 8051 Assembly,Python, SPI, UART, I2C, SDCC, Make, Git} \\
      {Hardware}&{\: Layout and Schematic Capture, Circuit Design, EAGLE, LTPSICE} \\
      {Lab Skills}&{\: Oscilloscopes, Logic Analyser, Soldering (SMD), Electronic Load, Function Generator} \\
    \end{tabular}
\end{table}
 \vspace{-16pt}
% %------RELEVANT COURSEWORK-------

%-----------EXPERIENCE-----------
\section{Experience}
\vspace{5pt}
  \resumeSubHeadingListStart
    \resumeSubheading
      {WCB Robotics Pvt Ltd}{Hyderabad, India}{Design Engineer}{July 2021 - May 2023}
      \vspace{-0.3cm}
      \begin{itemize}[leftmargin=0.01in]
        \item[]  \textbf{RF Transceiver board for Radio controller of robot}
            \begin{itemize}\setlength{\itemsep}{0pt}\setlength{\parskip}{0pt}\vspace{-0.3em}
                \item[$\bullet$]  Designed and implemented a high-range wireless transceiver with an impedance-matched PCB for a facade cleaning robot, achieving a line of sight range of 850 meters and reliable communication on high-rise buildings.
                \item[$\bullet$] Implemented a custom SPI driver for the CC2500 transceiver, offering an abstraction layer that simplified access to core functionality. Furthermore 
                \item[$\bullet$] Enhanced noise immunity of the system by implementing FHSS and antenna diversity in firmware.
            \end{itemize}
        \item[]  \textbf{Lithium Ion Battery Charger}
            \begin{itemize}\setlength{\itemsep}{0pt}\setlength{\parskip}{0pt}\vspace{-0.3em}
                \item[$\bullet$] Designed a compact Lithium Ion battery charger system by integrating a 4-layer PCB with the battery pack, allowing in-device charging without unplugging, while effectively managing thermal constraints. 
                \item[$\bullet$] Attained 250W/10A high-power charging using LT3763 IC, enabling a 21Ah battery to charged in 2.5 hours.
            \end{itemize}
        \item[]  \textbf{Pressure and Orientation Sensor Board}
            \begin{itemize}\setlength{\itemsep}{0pt}\setlength{\parskip}{0pt}\vspace{-0.3em}
                \item[$\bullet$] Designed a small form-factor(3cm x 3cm) sensor board for a glass building maintenance robot, vital for maintaining secure suction and precise orientation on vertical surfaces.
                \item[$\bullet$] Developed a two-PCB arrangement for differential pressure sensing, incorporating BMP388 pressure sensors on a rigid PCB(atmospheric pressure) and a 4cm Flex PCB(suction cup pressure).
                \item[$\bullet$] Wrote a fault-tolerant I2C driver for the sensors to detect sensor timeouts and report errors for timely remedial actions in the critical closed-loop control system, ensuring safe and reliable robot operation.
            \end{itemize}
        \item[]  \textbf{Data-Logging System for Housekeeping}
            \begin{itemize}\setlength{\itemsep}{0pt}\setlength{\parskip}{0pt}\vspace{-0.3em}
                \item[$\bullet$] Interfaced a Winbond W25N NAND Flash memory with the TMS570 microcontroller via SPI and developed user-friendly APIs for efficient data storage and retrieval, enhancing data-logging capabilities.
                \item[$\bullet$] Wrote a Python script for interpreting and processing retrieved data, improving data analysis and usability.
                %\item[$\bullet$] Later enhanced the project by implementing FatFS for TMS570, enabling the use of an SD Card for improved data storage and access in the form of CSV files.
            \end{itemize}           
    \end{itemize}
    \resumeSubHeadingListEnd
\vspace{-16pt}
\section{Projects}
  \resumeSubHeadingListStart
    \resumeSubheadingP
    {E-Ink Desk Clock}{November 2023 - December 2023}
    \begin{itemize}\setlength{\itemsep}{0pt}\setlength{\parskip}{0pt}\vspace{-0.2cm}
            \item[$\bullet$] Wrote a graphics library to display text and images on a Waveshare 4.2 Inch E-Ink Display.
            \item[$\bullet$] Designed a PCB and implemeted firmware for a a clock with user configurable alarms.      
        \end{itemize}
    \resumeSubheadingP
    {Atmel AT89C2051 Board bring-up}{August 2023 - November 2023}
    \begin{itemize}\setlength{\itemsep}{0pt}\setlength{\parskip}{0pt}\vspace{-0.2cm}
            \item[$\bullet$] Interfaced a '373 latch, NVSRAM with AT89C51R2 to demonstrate program execution from external memory.
            \item[$\bullet$] Set up a toolchain using SDCC, make and batchisp to compile, link and flash C programs.      
        \end{itemize}
    \resumeSubheadingP
    {Electrical Power Subsystem of a Cube-Sat}{March 2018 - August 2021}
    \begin{itemize}\setlength{\itemsep}{0pt}\setlength{\parskip}{0pt}\vspace{-0.2cm}
            \item[$\bullet$] Wrote I2C drivers for interfacing INA219 Power Monitor with the MSP430F5529 micro-controller to implement Maximum Power Point Tracking using the Perturb and Observe algorithm.{\href{https://github.com/parthkharade/MSP430F5529}{\{Github\}}}
        \end{itemize}
    \resumeSubHeadingListEnd
\vspace{-16pt}
\end{document}
