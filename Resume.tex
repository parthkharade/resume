% Resume in Latex
% Author : Parth Kharade
% chktex-file 8
% chktex-file 1

\documentclass[11pt]{article}
\usepackage{latexsym}
\usepackage[margin=1cm]{geometry}
\usepackage{titlesec}
\usepackage[usenames,dvipsnames]{color}
\usepackage{enumitem}
\usepackage[hidelinks]{hyperref}
\usepackage{tabularx}
\usepackage{fontawesome5}
\pagenumbering{gobble}
% \raggedbottom
\raggedright
\setlength{\tabcolsep}{0in}

% Sections formatting
\titleformat{\section}{
  \vspace{-4pt}\raggedright\large\bfseries
}{}{0em}{}[\color{black}\titlerule\vspace{-5pt}]

% Ensure that generate pdf is machine readable/ATS parsable
\input{glyphtounicode}
\pdfgentounicode=1

%-------------------------
% Custom commands

\newcommand{\resumeSubheadingEdu}[4]{
  \vspace{-2pt}\item
    \begin{tabular*}{1.0\textwidth}[t]{l@{\extracolsep{\fill}}r}
       {\large #1}&{\large #2} \\
       {#3} &{ #4} \\
    \end{tabular*}\vspace{-1pt}
}

\newcommand{\resumeSubheading}[3]{
  \vspace{-2pt}\item
    \begin{tabular*}{1.0\textwidth}[t]{l@{\extracolsep{\fill}}r}
      {\large \textbf{#1} \textbar{} #2}&{\large #3} \\
    \end{tabular*}\vspace{-10pt}
}


\newcommand{\resumeSubheadingP}[2]{
  \vspace{-2pt}\item
    \begin{tabular*}{1.0\textwidth}[t]{l@{\extracolsep{\fill}}r}
       \textbf{{\large #1}}&{#2} \\
    \end{tabular*}\vspace{-1pt}
}


\newcommand{\resumeSubHeadingListStart}{\begin{itemize}[leftmargin=0.0in, label={}]}
\newcommand{\resumeSubHeadingListEnd}{\end{itemize}}

\begin{document}
\setlength{\footskip}{4pt}
\begin{center}
  {\huge \scshape Parth Kharade} \\ \vspace{1pt}
  \small \raisebox{-0.1\height}\faPhone\ 303-725-7695{\hspace{0.1cm}}~\href{mailto:parthkharade99@gmail.com}{\raisebox{-0.2\height}\faEnvelope\ \underline{parthkharade99@gmail.com}}~{\hspace{0.1cm}}\href{https://www.linkedin.com/in/parth-k-081287184/}{\raisebox{-0.2\height}\faLinkedin\ \underline{linkedin.com/in/parthkharade}}
  \vspace{-10pt}
\end{center}


%-----------EDUCATION-----------

\section {Education}
\resumeSubHeadingListStart
\resumeSubheadingEdu
{University of Colorado Boulder}{August 2023 - May 2025}
{Master's in Embedded Systems Engineering}{3.72/4.00}
\vspace{-0.1in}
\resumeSubheadingEdu
{Birla Institute of Technology and Science, Pilani}{August 2017 - May 2021}
{Bachelor's in Electrical and Electronics Engineering}{7.85/10.0}
\resumeSubHeadingListEnd
\vspace{-16pt}
\section{Skills and Coursework}
\vspace{-4mm}
\begin{table}[!htb]
  \begin{tabular} { m{3.1cm} | m{15cm} }
    % {Micro-controllers}   & {\: STM32 (Arm Cortex-M4F), NXP (Arm Cortex-M0+), TI-MSP, TI-C2000, 8051}                  \\
    { Languages }         & {\: C, C++, 8051 Assembly, Python, Verilog, VHDL}               \\
    {Tools}               & {\: Make, Bazel, Git, GTest, ProtoBuf, Quartus Prime}               \\
    {Peripherals}         & {\: GPIO, SPI, UART, I2C, ADC, DAC, DMA, Timers}                                          \\
    % {Hardware}          & {\: Layout and Schematic Capture, Circuit Design, EAGLE, LTSpice}                        \\
    {Hardware Skills}          & {\: Oscilloscopes, Logic Analyser, Soldering (SMD), Electronic Load, Function Generator} \\
    {Coursework}          & {\: IoT Embedded Firmware, Embedded Linux Design, Programmable Logic Design}                \\
    {}                    & {\: Real-Time Embedded Systems, Embedded Systems Design}
    % {}&{\: Development}
  \end{tabular}
\end{table}
\vspace{-16pt}
%-----------EXPERIENCE-----------
\section{Experience}
\resumeSubHeadingListStart
\resumeSubheading
{Carbon3D}{Software Engineering Intern}{May 2024 - August 2024}
\vspace{-0.3cm}
\begin{itemize}[leftmargin=0.3in]\vspace{-0.3em}\setlength{\itemsep}{0pt}
  \item[$\bullet$]Wrote application level code and tests in GoogleTest to make the internal state-machines of the printer more robust. This completely eliminated unexpected state-transitions and random behaviour. 
  \item[$\bullet$] Built a firmware update manager in Python to provide a consistent interface for managing printer firmware. 
\end{itemize}
%---------------------------------------------------------------------%
\resumeSubheading
{WCB Robotics Pvt Ltd}{Design Engineer}{July 2021 - May 2023}
\vspace{-0.4cm}
\begin{itemize}[leftmargin=0.01in]\setlength\itemsep{-5pt}
  \item[]  \textbf{RF Transceiver}
    \begin{itemize}\setlength{\itemsep}{0pt}\setlength{\parskip}{0pt}\vspace{-0.3em}
      \item[$\bullet$] Built a SPI driver for the CC2500 transceiver that simplified access to core functionality. Enhanced noise immunity of the system by implementing FHSS and antenna diversity in firmware.
      \item[$\bullet$] Designed a high-range wireless transceiver with an impedance-matched PCB for a facade cleaning robot, achieving a line of sight range of 850 meters and reliable communication on high-rise buildings.
    \end{itemize}
  \item[]  \textbf{Pressure and Orientation Sensing System}
    \begin{itemize}\setlength{\itemsep}{0pt}\setlength{\parskip}{0pt}\vspace{-0.3em}
      \item[$\bullet$] Wrote a fault-tolerant I2C driver for the sensors to detect sensor timeouts and report errors for timely remedial actions in the critical closed-loop control system, ensuring safe and reliable robot operation.
        % \item[$\bullet$] Developed a two-PCB (rigid +flex) arrangement for orientation and differential pressure sensing, with BMP388 and BNO055 sensors. This was vital for maintaining secure suction and precise orientation on vertical surfaces.
    \end{itemize}
  % \item[]  \textbf{Lithium Ion Battery Charger}
  %   \begin{itemize}\setlength{\itemsep}{0pt}\setlength{\parskip}{0pt}\vspace{-0.3em}
  %     \item[$\bullet$] Designed a compact Li-ion battery charger system by integrating a 4-layer PCB with the battery pack, allowing in-device charging without unplugging, while effectively managing thermal constraints. 
  %     \item[$\bullet$] Attained 250W/10A high-power charging using LT3763 IC, enabling a 21Ah battery to charged in 2.5 hours.
  %   \end{itemize}

  \item[]  \textbf{Data-Logging System for Housekeeping}
    \begin{itemize}\setlength{\itemsep}{0pt}\setlength{\parskip}{0pt}\vspace{-0.3em}
      \item[$\bullet$] Interfaced a Winbond W25N NAND-Flash memory with a TMS570 microcontroller via SPI and wrote user-friendly APIs for efficient data storage and retrieval, enhancing data-logging capabilities.
        % \item[$\bullet$] Wrote a Python script for interpreting and processing retrieved data, improving data analysis and usability.
    \end{itemize}
\end{itemize}
%---------------------------------------------------------------------%
\resumeSubheading
{CU Boulder}{Teaching Assistant - Embedded System Design}{January 2024 - Present}
\vspace{-0.3cm}
\begin{itemize}[leftmargin=0.3in]\vspace{-0.3em}
  \item[$\bullet$]Responsible for supporting student learning by helping students debug complex hardware and software issues, conducting code reviews,  administering lab sign-offs and grading assignments.
\end{itemize}
\vspace{-0.2cm}
\resumeSubHeadingListEnd
%---------------------------------------------------------------------%
\vspace{-16pt}

\section{Projects}
\resumeSubHeadingListStart

\resumeSubheadingP
{E-Paper Desk Clock \textbar{ } Bare-Metal Embedded \href{https://github.com/parthkharade/Eink-DeskClock}{\small{\textbar{ }\underline{Link} }}}{November 2023 - December 2023}
\begin{itemize}\setlength{\itemsep}{0pt}\setlength{\parskip}{0pt}\vspace{-0.2cm}
  \item[$\bullet$] Built a desk-clock using an E-Paper display. Designed a PCB and wrote low-level peripheral drivers for STM32. 
  \item[$\bullet$] Developed a custom graphics library  to allow the user to render bit mapped text and images on the display.
  % \item[$\bullet$] Implemented the Model-View-Controller design pattern using a state machine to allow the user to interact with the clock. The user could set date and time as well as configure 4 independant alarms.
\end{itemize}
\vspace{-0.3cm}

\resumeSubheadingP
{Boulder Bash \textbar{ } IoT Embedded Firmware}{April 2024 - May 2024}
\begin{itemize}\setlength{\itemsep}{0pt}\setlength{\parskip}{0pt}\vspace{-0.2cm}
  \item[$\bullet$] Developed 'Boulder Bash', a retro arcade console game using Bluetooth Low Energy (BLE) with custom GATT services for data exchange between a energy-efficient wireless joystick and console with a dot matrix display. 
\end{itemize}
\vspace{-0.3cm}

\resumeSubheadingP
{NIOS II synthesis on MAX10 FPGA \textbar{ } FPGA Design}{September 2024 - Present}
\begin{itemize}\setlength{\itemsep}{0pt}\setlength{\parskip}{0pt}\vspace{-0.2cm}
  \item[$\bullet$] Designed a NIOS II system using Qsys and synthesized it on a MAX10 FPGA. Understood critical steps in hardware development flow such as timing closure, gate-level simulations and soft IP programming. 
\end{itemize}
\vspace{-0.3cm}

\resumeSubheadingP
{Spotify Dock \textbar{ } Embedded Linux \href{https://github.com/cu-ecen-aeld/final-project-parthkharade/wiki/Project-Overview}{\small{\textbar{ }\underline{Link} }}}{April 2024 - May 2024}
\begin{itemize}\setlength{\itemsep}{0pt}\setlength{\parskip}{0pt}\vspace{-0.2cm}
  \item[$\bullet$] Built a Spotify Dock, a device capable of displaying and controlling live Spotify playback using bash scripts to interface with the Spotify API. Built a custom Buildroot image and wrote I2C and GPIO drivers for RPi.
\end{itemize}
\vspace{-0.3cm}
%
% \resumeSubheadingP
% {MIPS Processor \textbar{ } Computer Architecture}{March 2020 - May 2020}
% \begin{itemize}\setlength{\itemsep}{0pt}\setlength{\parskip}{0pt}\vspace{-0.2cm}
%   \item[$\bullet$] Designed a 5-stage MIPS pipelined processor in Verilog capable of executing \emph{sub}, \emph{add}, \emph{or}, \emph{and} and \emph{beq} instructions. The pipeline was also able to detect and remedy structural, control and data hazards.
% \end{itemize}
\vspace{-0.3cm}
%
% \resumeSubheadingP
% {Electrical Power Subsystem of a Cube-Sat \href{https://github.com/parthkharade/MSP430F5529}{\small{\textbar{ }\underline{Link} }}}{March 2018 - August 2021}
% \begin{itemize}\setlength{\itemsep}{0pt}\setlength{\parskip}{0pt}\vspace{-0.2cm}
%   \item[$\bullet$] Wrote I2C drivers for interfacing INA219 Power Monitor with the MSP430F5529 microcontroller to implement Maximum Power Point Tracking using the Perturb and Observe algorithm.
% \end{itemize}
% \vspace{-0.3cm}

% \resumeSubheadingP
% {Microchip AT89C51RC2 Board bring-up \href{https://drive.google.com/drive/folders/1rbtfpOdMc_ohnfu15VS3bPxNDx-BoWwA?usp=drive_link}{\small{\textbar{ }\underline{Link}}}}{August 2023 - November 2023}
% \begin{itemize}\setlength{\itemsep}{0pt}\setlength{\parskip}{0pt}\vspace{-0.2cm}
%   \item[$\bullet$] Built an AT89C51RC2 development board from scratch, interfacing various peripherals such as external RAM, LCD, GPIO expander, DAC, and ADC, and enabled in-system programming using a MAX232 IC.
%   % \item[$\bullet$] Wrote terminal-based, menu-driven, user-interactive programs to allow the user to interact with and configure on-chip as well as off-chip peripherals.
% \end{itemize}
% \vspace{-0.3cm}


\resumeSubHeadingListEnd
\end{document}
