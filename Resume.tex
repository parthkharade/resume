\documentclass[11pt]{article}
% Resume in Latex
% Author : Jake Gutierrez,Parth Kharade
% Based off of: https://github.com/sb2nov/resume
% License : MIT

% chktex-file 8
% chktex-file 1

\usepackage{latexsym}
\usepackage[empty]{fullpage}
\usepackage{titlesec}
\usepackage{marvosym}
\usepackage[usenames,dvipsnames]{color}
\usepackage{verbatim}
\usepackage{enumitem}
\usepackage[hidelinks]{hyperref}
\usepackage{fancyhdr}
\usepackage[english]{babel}
\usepackage{tabularx}
\usepackage{fontawesome5}
\usepackage{multicol}
\setlength{\multicolsep}{-3.0pt}
\setlength{\columnsep}{-1pt}
\input{glyphtounicode}
\usepackage{array}

\pagestyle{fancy}
\fancyhf{} % clear all header and footer fields
\fancyfoot{}
\renewcommand{\headrulewidth}{0pt}
\renewcommand{\footrulewidth}{0pt}

% Adjust margins
\addtolength{\oddsidemargin}{-0.6in}
\addtolength{\evensidemargin}{-0.5in}
\addtolength{\textwidth}{1.19in}
\addtolength{\topmargin}{-.78in}
\addtolength{\textheight}{1.4in}

\urlstyle{same}

\raggedbottom\raggedright\setlength{\tabcolsep}{0in}

% Sections formatting
\titleformat{\section}{
  \vspace{-4pt}\scshape\raggedright\large\bfseries
}{}{0em}{}[\color{black}\titlerule\vspace{-5pt}]

% Ensure that generate pdf is machine readable/ATS parsable
\pdfgentounicode=1

%-------------------------
% Custom commands
\newcommand{\resumeItem}[1]{
  \item\small{
    {#1\vspace{-2pt}}
  }
}

\newcommand{\classesList}[4]{
    \item\small{
        {#1 #2 #3 #4\vspace{-2pt}}
  }
}

\newcommand{\resumeSubheading}[4]{
  \vspace{-2pt}\item
    \begin{tabular*}{1.0\textwidth}[t]{l@{\extracolsep{\fill}}r}
       {\large \textbf{#1}}&{\large #2} \\
      \textbf{#3}&{ #4} \\
    \end{tabular*}\vspace{-1pt}
}

\newcommand{\resumeSubheadingP}[2]{
  \vspace{-2pt}\item
    \begin{tabular*}{1.0\textwidth}[t]{l@{\extracolsep{\fill}}r}
       \textbf{{\large #1}}&{#2} \\
    \end{tabular*}\vspace{-1pt}
}

\newcommand{\resumeSubSubheading}[2]{
    \item
    \begin{tabular*}{1.0\textwidth}{l@{\extracolsep{\fill}}r}
      \textit{\small#1} & \textit{ #2} \\
    \end{tabular*}\vspace{-7pt}
}

\newcommand{\resumeProjectHeading}[2]{
    \item
    \begin{tabular*}{1.0\textwidth}{l@{\extracolsep{\fill}}r}
      \small#1 &  {\small #2}\\
    \end{tabular*}\vspace{-7pt}
}

\newcommand{\resumeSubItem}[1]{\resumeItem{#1}\vspace{-4pt}}

\renewcommand\labelitemi{$\vcenter{\hbox{\tiny$\bullet$}}$}
\renewcommand\labelitemii{$\vcenter{\hbox{\tiny$\bullet$}}$}

\newcommand{\resumeSubHeadingListStart}{\begin{itemize}[leftmargin=0.0in, label={}]}
\newcommand{\resumeSubHeadingListEnd}{\end{itemize}}
\newcommand{\resumeItemListStart}{\begin{itemize}}
\newcommand{\resumeItemListEnd}{\end{itemize}\vspace{-5pt}}
\newcommand{\project}[3]{\texorpdfstring{$\bullet$}{}\hspace{4pt} {#1}\texorpdfstring{\hfill}{}\emph{#2}\texorpdfstring{\\}{}\hspace*{8pt}\emph{#3}}

\begin{document}
\setlength{\footskip}{4.1pt}
\begin{center}
  {\huge \scshape Parth Kharade} \\ \vspace{1pt}
  \small \raisebox{-0.1\height}\faPhone\ 303-725-7695{\hspace{0.1cm}}~\href{mailto:parthkharade99@gmail.com}{\raisebox{-0.2\height}\faEnvelope\ \underline{parthkharade99@gmail.com}}~{\hspace{0.1cm}}\href{https://www.linkedin.com/in/parth-k-081287184/}{\raisebox{-0.2\height}\faLinkedin\ \underline{linkedin.com/in/parthkharade}}
  \vspace{-10pt}
\end{center}


%-----------EDUCATION-----------
\section {Education}
\resumeSubHeadingListStart
\resumeSubheading
{University of Colorado Boulder}{August 2023 - May 2025}
{Master's in Embedded Systems Engineering}{3.80/4.00}
\vspace{-0.1in}
\resumeSubheading
{Birla Institute of Technology and Science, Pilani}{August 2017 - May 2021}
{Bachelor's in Electrical and Electronics Engineering}{7.85/10.0}
\resumeSubHeadingListEnd
\vspace{-16pt}
\section{Skills and Coursework}
\vspace{-4mm}
\begin{table}[!htb]
  \begin{tabular} { m{3.1cm} | m{16cm} }
    {Micro-controllers} & {\: STM32 (Arm Cortex-M4F), NXP (Arm Cortex-M0+), MSP430, TMS320, 8051}                  \\
    {Firmware}          & {\: Embedded C, C++, 8051 Assembly, Python, Bash, Make, Bazel, Git, GTest}               \\
    {Peripherals}       & {\: GPIO, SPI, UART, I2C, ADC, DAC, DMA,Timers}                                          \\
    {Hardware}          & {\: Layout and Schematic Capture, Circuit Design, EAGLE, LTSpice}                        \\
    {Lab Skills}        & {\: Oscilloscopes, Logic Analyser, Soldering (SMD), Electronic Load, Function Generator} \\
    {Coursework}        & {\: Embedded System Design, IoT Embedded Firmware, Embedded Linux Design}                \\
    % {}&{\: Development}
  \end{tabular}
\end{table}
\vspace{-16pt}
% %------RELEVANT COURSEWORK-------

%-----------EXPERIENCE-----------
\section{Experience}
\vspace{5pt}
\resumeSubHeadingListStart
\resumeSubheading
{Carbon3D}{Redwood City, California}{Softare Engineering Intern}{May 2024 - August 2024}
\vspace{-0.3cm}
\begin{itemize}[leftmargin=0.3in]\vspace{-0.3em}\setlength{\itemsep}{0pt}
  \item[$\bullet$]Wrote application level code and tests to make the internal state-machines of the printer more robust and less prone to critical errors.
  \item[$\bullet$] Developed a firmwware update manager in Python to provide a hassle-free experience in managing firmware for the printer's microcontrollers
\end{itemize}
\vspace{-0.2cm}
\resumeSubheading
{University of Colorado Boulder}{Boulder, Colorado}{Graduate Teaching Assistant - Embedded System Design }{January 2024 - Present}
\vspace{-0.3cm}
\begin{itemize}[leftmargin=0.3in]\vspace{-0.3em}
  \item[$\bullet$]Responsible for supporting student learning by helping students debug complex hardware and software issues, conducting code reviews,  administering lab sign-offs and grading assignments.
\end{itemize}
\vspace{-0.2cm}
\resumeSubheading
{WCB Robotics Pvt Ltd}{Hyderabad, India}{Design Engineer}{July 2021 - May 2023}
\vspace{-0.3cm}
\begin{itemize}[leftmargin=0.01in]
  \item[]  \textbf{RF Transceiver Board for Radio Controller of Robot}
        \begin{itemize}\setlength{\itemsep}{0pt}\setlength{\parskip}{0pt}\vspace{-0.3em}
          \item[$\bullet$] Designed a high-range wireless transceiver with an impedance-matched PCB for a facade cleaning robot, achieving a line of sight range of 850 meters and reliable communication on high-rise buildings.
          \item[$\bullet$] Built a SPI driver for the CC2500 transceiver that simplified access to core functionality. Enhanced noise immunity of the system by implementing FHSS and antenna diversity in firmware.
        \end{itemize}
  \item[]  \textbf{Pressure and Orientation Sensor Board}
        \begin{itemize}\setlength{\itemsep}{0pt}\setlength{\parskip}{0pt}\vspace{-0.3em}
          \item[$\bullet$] Developed a two-PCB (rigid +flex) arrangement for orientation and differential pressure sensing, with BMP388 and BNO055 sensors. This was vital for maintaining secure suction and precise orientation on vertical surfaces.
          \item[$\bullet$] Wrote a fault-tolerant I2C driver for the sensors to detect sensor timeouts and report errors for timely remedial actions in the critical closed-loop control system, ensuring safe and reliable robot operation.
        \end{itemize}
  \item[]  \textbf{Data-Logging System for Housekeeping}
        \begin{itemize}\setlength{\itemsep}{0pt}\setlength{\parskip}{0pt}\vspace{-0.3em}
          \item[$\bullet$] Interfaced a Winbond W25N NAND-Flash memory with a TMS570 microcontroller via SPI and wrote user-friendly APIs for efficient data storage and retrieval, enhancing data-logging capabilities.
          \item[$\bullet$] Wrote a Python script for interpreting and processing retrieved data, improving data analysis and usability.
        \end{itemize}
\end{itemize}
\resumeSubHeadingListEnd
\vspace{-16pt}
\section{Projects}
\resumeSubHeadingListStart
\resumeSubheadingP
{E-Paper Display Based Desk Clock \href{https://github.com/parthkharade/Eink-DeskClock}{\small{\textbar{ }\underline{Link} }}}{November 2023 - December 2023}
\begin{itemize}\setlength{\itemsep}{0pt}\setlength{\parskip}{0pt}\vspace{-0.2cm}
  \item[$\bullet$] Developed a custom graphics library  to allow the user to render bit mapped text and images on the display.
  \item[$\bullet$] Implemented the Model-View-Controller design pattern using a state machine to allow the user to interact with the clock.
\end{itemize}
\vspace{-0.3cm}
\resumeSubheadingP
{Microchip AT89C51RC2 Board bring-up \href{https://drive.google.com/drive/folders/1rbtfpOdMc_ohnfu15VS3bPxNDx-BoWwA?usp=drive_link}{\small{\textbar{ }\underline{Link}}}}{August 2023 - November 2023}
\begin{itemize}\setlength{\itemsep}{0pt}\setlength{\parskip}{0pt}\vspace{-0.2cm}
  \item[$\bullet$] Built an AT89C51RC2 development board from scratch, interfacing various peripherals such as external RAM, LCD, GPIO expander, DAC, and ADC, and enabled in-system programming using a MAX232 IC.
  \item[$\bullet$] Wrote terminal-based, menu-driven, user-interactive programs to allow the user to intertact with and configure on-chip as well as off-chip peripherals.
\end{itemize}
\vspace{-0.3cm}
\resumeSubheadingP
{Electrical Power Subsystem of a Cube-Sat \href{https://github.com/parthkharade/MSP430F5529}{\small{\textbar{ }\underline{Link} }}}{March 2018 - August 2021}
\begin{itemize}\setlength{\itemsep}{0pt}\setlength{\parskip}{0pt}\vspace{-0.2cm}
  \item[$\bullet$] Wrote I2C drivers for interfacing INA219 Power Monitor with the MSP430F5529 microcontroller to implement Maximum Power Point Tracking using the Perturb and Observe algorithm.
\end{itemize}
\vspace{-0.3cm}

\resumeSubheadingP
{Electrical Power Subsystem of a Cube-Sat \href{https://github.com/parthkharade/MSP430F5529}{\small{\textbar{ }\underline{Link} }}}{March 2018 - August 2021}
\begin{itemize}\setlength{\itemsep}{0pt}\setlength{\parskip}{0pt}\vspace{-0.2cm}
  \item[$\bullet$] Wrote I2C drivers for interfacing INA219 Power Monitor with the MSP430F5529 microcontroller to implement Maximum Power Point Tracking using the Perturb and Observe algorithm.
\end{itemize}

\resumeSubHeadingListEnd
\vspace{-16pt}
\end{document}